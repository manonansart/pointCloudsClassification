\subsection{Compte-rendu de la réunion avec les tuteurs projet, du 26/01/15}

Il a été convenu avec les tuteurs pédagogiques que les réunions se tiendraient toutes les deux semaines. \\

Les données que nous avons sont plusieurs imagettes (frames) de chaque classes (une dizaine d'objets). Ce sont des sortes de captures d'écran d'une vidéo à un instant donné, mais en trois dimensions. Par exemple, nous pourrions avoir 200 imagettes pour l'objet vélo, 150 imagettes pour l'objet voiture, 150 pour l'objet piéton, etc.  \\

Soit$ X_{i}$ la ième imagette, on a :

$  X_{i} \in \reels^{n_{i}*d} $, avec $10^{3} \leq n_{i} \leq 72.10^{3}$, et  $3 \leq d \leq 5$
En effet, chaque imagette a ses coordonnées x, y, z au minimum, et éventuellement la profondeur et la couleur (RGB par exemple). \\ 

A noter aussi que les objets sont tous déjà séparés. \\

Nous avons aussi brièvement parlé de spin image, pour pouvoir avoir des images avec taille prédéfinie, et donc résoudre le problème de dimension de l'image. \\

\textbf{A réaliser pour le 09/02/15 : }
Il faut réussir à prendre en main la bibliothèque C++ PointClouds.\\
Il faut parvenir à utiliser quelques méthodes, par exemple des méthodes de clustering. Nous allons devoir regrouper les objets par rapport à leurs labels, autrement dit organiser les données par classe. Il faut importer/exporter les fichiers en .pcd (ou .pcl), afin de pouvoir les lire sous Matlab ou via un Viewer de Pointclouds.