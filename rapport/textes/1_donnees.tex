\section{Extraction des données}

	Les données sont extraites à partir de ROS. Il s'agit d'un outil qui sert à traiter les .tm, qui sont des vidéos en 3D, pour en extraire des nuages de points au format pcd. Ici, les vidéos que nous avons sont filmées depuis un véhicule dans une zone urbaine.\\

	Vu qu’il y a des erreurs dans la compilation du C++ pour lancer ROS, M. Guerrero nous a fourni directement les archives contenant les pcd.\\

\section{Description des données}

	Les données que nous avons sont une multitude imagettes (frames) de chaque classe. Ce sont des sortes de captures d'écran d'une vidéo à un instant donné, mais en trois dimensions.\\

	Au total, nous avons les classes suivantes : 
	% à compléter classe + nombre d'images de la classe

	Soit$ X_{i}$ la ième imagette, on a :

	$  X_{i} \in \reels^{n_{i}*4} $, avec 50 $\leq n_{i} \leq 300$
	En effet, chaque imagette a ses coordonnées x, y, z et une valeur d'intensité. \\ 