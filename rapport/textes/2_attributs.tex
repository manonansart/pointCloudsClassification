Afin de mettre en place nos classifieurs pour identifier les nuages de points 3D, nous avons tout d'abord dû extraire plusieurs métriques nous permettant de représenter nos images de points. Nous prenions pour cela en entrée un fichier par nuage de points, comportant pour chaque point ses coordonnées (x, y et z) et son intensité

\section{Attributs utilisés}
	\subsection{Métriques statistiques basiques}
		Dans un premier temps, nous avons extrait des métriques statistiques très simples basées sur l'intensité des points. Nous avons pour chaque nuage de points calculé la moyenne, la variance, le minimum et le maximum des intensités.\\

		En effet, ces valeurs changent selon les classes. On peut le voir dans le tableau suivant, présentant les moyennes des trois attributs sur des nuages de points de type voitures et background :\\

		\begin{center}
			\begin{tabular}{|l||c|c|c|c|}
			  \hline
			  & Minimum & Maximum & Moyenne & Variance \\
			  \hline
			  Voitures & 34.5 & 249 & 120 & 3500 \\
			  Background & 48.7 & 196 & 125 & 1548 \\
			  \hline
			\end{tabular}
		\end{center}


		Cela nous montre que ces attributs peuvent présenter des différences significatives selon les classe, il est donc intéresssant de prendre en compte ces métriques pour la classification.

	\subsection{Bounding-box}
		La \emph{bounding-box} permet de connaître la forme de l'objet en 3D pour distinguer des objets de catégorie différentes.

		\begin{figure}[H]
			\centering
			\includegraphics[scale=0.6]{images/bounding-box.png}
			\caption{Exemple de \emph{bounding-box}}
			\label{fig:image}
		\end{figure}
		% Source : http://www.mathworks.com/matlabcentral/fileexchange/screenshots/2122/original.jpg

		La \emph{bounding-box} représente l'amplitude du nuage de points selon les trois dimensions. Pour chaque dimension, nous aurons donc un attribut qui correspond à la différence entre le maximum de cette coordonnée et le minimum de cette coordonnée, pour chaque nuage de points.\\

		Afin de pouvoir calculer la \emph{bounding-box}, il nous a tout d'abord fallu  "redresser" le nuage de points selon l' axe $z$ (nous considérons que le terrain est plat et qu'il n'y a donc pas besoin de redresser l'image selon $x$ et $y$).
		Nous avons donc calculé l'angle $\theta$ de la rotation, grâce à la formule:

		\[ tan(2 \theta) = 2 * \frac{\bar{x}\bar{y}}{\bar{x}^2 - \bar{y}^2} \]
		\[ \theta = \frac{1}{2} * arctan(2 \frac{\bar{x}\bar{y}}{\bar{x}^2 - \bar{y}^2}) \]

		avec \[ \bar{x} = \frac{1}{n} * \sum_{i =1}^n{x_i} \]
		\[ \bar{y} = \frac{1}{n} * \sum_{i =1}^n{y_i} \]

		Une fois que nous avons trouvé l'angle de rotation, il faut appliquer la rotation au nuage de points. Pour cela on construit la matrice $R_z$ telle que :

		\[\begin{bmatrix}
		   cos(\theta) & -sin(\theta) & 0 \\
		sin(\theta) & cos(\theta) & 0 \\
		0 & 0 & 1
		\end{bmatrix}\]

		On calcule ensuite le nouveau nuage de point redressé avec la formule:
		\[ P_T = R_z * P_0\]
		où $P_0$ est le nuage de points d'origine.

		On peut ensuite calculer les 3 attributs de la \emph{bounding-box} sur ce nouveau nuage de points.


	\subsection{Scatter-ness, linear-ness et surface-ness}
		D'autres attributs intéressants à étudier dans le cadre de la reconnaissance d'objets 3D sont la \emph{scatter-ness}, la \emph{linear-ness} et la \emph{surface-ness}, qui traduisent la forme de l'objet. 

		Pour les calculer, on calcule tout d'abord la matrice de covariance des coordonnées des points, puis on calcule les valeurs propres $\lambda_0$, $\lambda_1$ et $\lambda_2$ de la matrice de covariance. \\

		On s’intéresse ensuite à la différence entre ces valeurs : 
		\begin{itemize}
			\item Si ces trois valeurs sont similaires ($\lambda_0 \approx \lambda_1 \approx \lambda_2$), les points sont répartis de manière égale selon les trois dimensions, comme pour une sphère. On dit qu'ils sont dispersés et on parle de \emph{scatter-ness}.
			\item Si au contraire une valeur est beaucoup plus grande que les deux autres ($\lambda_0 \gg \lambda_1 \approx \lambda_2$), cela signifie que les points sont principalement répartis sur une seule dimension, ils forment une sorte de ligne. On parle alors de \emph{linear-ness}.
			\item Si une valeur est beaucoup plus petite que les deux autres ($\lambda_0 \approx \lambda_1 \gg \lambda_2$), cela signifie que les points sont répartis sur deux dimensions, et forment quasiment une surface. On parle alors de \emph{surface-ness}.\\
		\end{itemize}

		Nous allons donc nous intéresser aux attributs suivants :
		\begin{itemize}
			\item $\lambda_0$, qui représente la \emph{scatter-ness};
			\item $\lambda_0 - \lambda_1$, qui représente la \emph{linear-ness};
			\item $\lambda_1 - \lambda_2$, qui représente la \emph{surface-ness};
		\end{itemize}

\section{Attribut étudié : les spin-images}

	La spin image est l’image du nuage 3D prise en un point, perpendiculairement à la normale en ce point (donc selon le plan tangent). Le but est de prendre est de prendre les meilleures spin images pour chaque nuage de point car c’est sur ces images 2D que nous pouvons réaliser une classification.




\section{Autres attributs possibles}
