\subsection{Lecture et visualisation d'un fichier .pcd}

\emph{Cloud Viewer} étant trop "simple" pour notre utilisation, nous nous sommes penchés sur un autre viewer. \emph{PCLVisualizer}, proposé dans le tutoriel \url{http://pointclouds.org/documentation/tutorials/pcl_visualizer.php#pcl-visualizer} nous a semblé adéquat. \\

Nous avons le même problème qu'avec \emph{Cloud Viewer}, à savoir que notre nuage de points n'est toujours pas centré par rapport aux axes ; or, le zoom ne se fait que par rapport au centre des axes, donc nous ne pouvons pas voir le nuage précisément. Il faut donc que l'on réussisse à centrer nos données avant de les visualiser. En remarquant par exemple que les coordonnées x du fichier "segmented\_0segment1.pcd" ont une moyenne d'environ -8, en rajoutant 8 à toutes ces coordonnées on arrive à visualiser correctement le nuage de points (voir visualizer 2). \\

A noter aussi que les méthodes proposées par \emph{PCLVisualizer}, mis à part le viewer de base \textit{simpleVis}, sont à utiliser avec des données en XYZRGB. Nous avons essayé de lire le fichier "segmented\_0segment1.pcd" en le forçant à être lu en XYZRGB et en utilisant la méthode \textit{rgbVis}, nous n'avons pas obtenu d'erreurs à la compilation mais rien n'apparaissait à l'écran. Nos données semblent être des données uniquement XYZ. \\

\subsection{Description des données}

Nous possédons un jeu de 3173 imagettes, toutes au format .pcd, divisées en plusieurs types d'objets, que nous appelerons classes. Nous avons les classes suivantes : \\

\begin {itemize}
\item 0 : Vélo (953 imagettes);
\item 1 : -- (419 imagettes) ;
\item 2 : -- (393 imagettes) ;
\item 3 : -- (354 imagettes) ;
\item 4 : -- (350 imagettes) ;
\item 5 : -- (340 imagettes) ;
\item 6 : -- (330 imagettes) ;
\item 7 : -- (34 imagettes).
\end{itemize}

\subsection{Essais de méthodes de clustering}

Dans le dossier essai\_segmentation, nous avons testé une méthode de segmentation plane afin de mieux appréhender la librairie PointClouds. \\
Des points sont d'abord générés de façon à former un plan, et d'autres sont volontairement placés à l'écart. Le programme cherche ensuite les points qui forment un plan, et calcule ses paramètres. \\
Nous avons affiché le jeu de données, mais la méthode simpleVis décrite précédemment ne permet pas de colorer différemment les points. \\

De même, dans le dossier essai\_segmentation\_2, nous avons testé une méthode de segmentation par croissance de régions. A noter cependant que cela ne fonctionne pas avec nos fichiers .pcd (core dump), nous avons dû utiliser le fichiers .pcd donné dans le tutoriel associé.

